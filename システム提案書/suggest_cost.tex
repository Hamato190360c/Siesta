\documentclass[a4paper,dvipdfmx]{jsarticle}
\usepackage[dvips]{graphicx}
\usepackage{epsbox}
\usepackage{here}
\usepackage{float}
\begin{document}
\section{システム化にかかる費用とその効果}
システム化にかかる費用の概算は次の通りです。
\begin{table}[H]
  \caption{システム化にかかる費用}
  \begin{center}
    \begin{tabular}{|c|c|c|c|c|} \hline
      項目&単価(円)&数量&金額(円)&備考 \\ \hline
      Raspberry~pi&6,000&3台&18,000& \\ \hline
      USBカメラ&3,000&1台&3,000& \\ \hline
      サーバ用PC&150,000&1台&150,000& \\ \hline
      位置センサ&3,000&3台&9,000& \\ \hline
      保守・管理費& & & 減価償却期間& \\ \hline
      システム開発人件費&20,000&60日&&工数内訳 8人×60日 \\ \hline
      \multicolumn{3}{|c|}{合計}&& \\ \hline
    \end{tabular}
  \end{center}
\end{table}
システム化による効果の資産を以下に示します。前提条件として、入場者が本システムの使用が可能な端末を所持しており、動物園がネットワーク管理下にあると想定します。この場合、動物園来場者の増加、動物園来場者の満足度の向上などによるイメージアップが見込まれます。

利益と費用の比較

来場者は、宣伝や広告によってアプリをダウンロードし、アプリの機能を使用することで、野市動物園を効率よく楽しむことができます。
\section{本システム提案のアピールポイント}
本システム提案におけるアピールポイントについて説明します。
\begin{description}
\item[(1)] 動物園来場者に対して、動物園を効率的に楽しんで頂くためのAR搭載型アプリケーションシステムです。現在地や目的地をマップで表示することで、来場者の園内の移動効率を向上させます。
\item[(2)] 本アプリケーションは、英語、中国語に対応しているため、外国人来場者の方でも動物園をお楽しみ頂けます。
\item[(3)] 本アプリケーションでは、動物付近に設置されているwebカメラを中継し、園内の動物の様子を閲覧することができます。
\item[(4)] (1),(2),(3)のような多様な機能を実現させることによって、動物園の利用効率の向上を可能にさせます。その結果として来場者の増加や動物園のイメージアップが見込まれます。
\end{description}
  \section{用語の定義}
  本提案書では、次の通りに用語を定義します。
  \begin{itemize}
  \item Raspberry~Pi: ARMプロセッサを搭載したシングルポートコンピュータ
    \item AR: Augmented~Reality(拡張現実)の略。現実世界の映像に対し、位置情報などのデータや実際に存在しない情報をCGと重ねて表示させる手法。
\end{document}
