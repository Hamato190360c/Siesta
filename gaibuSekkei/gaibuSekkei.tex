\documentclass[a4j]{jarticle}

\usepackage[dvipdfmx]{graphicx}
\usepackage[dvipdfmx]{color}
\usepackage{epsbox}
\usepackage{url}
\usepackage{here}
\usepackage{ascmac}

\setlength{\headsep}{-5mm}
\setlength{\oddsidemargin}{0mm}
\setlength{\textwidth}{165mm}
\setlength{\textheight}{230mm}
\setlength{\footskip}{20mm}

\title{
\vspace{30mm}
{\bf システム提案書} 
\\
\vspace{5mm}
{\bf }
\vspace{90mm}
}

\author{
\vspace{5mm}
Siesta \\
}

%date{
%平成29年4月17日
%}

\begin{document}
\maketitle

\newpage

\tableofcontents

\newpage

\section{はじめに}
近年では年々の入園者数が減少・大きな赤字など問題を抱えている動物園が多数存在しています.
のいち動物公園様の有料入園者数も年々減少傾向であることが確認できました. 
そこで, 弊社はARを用いて入園者に動物園を楽しんで頂くシステム「ZoologicAR」を開発しました\cite{bibi1}.

本書では,本システムの機能について, システム要件を実現するためのシステム外部から見た設計条件を規定します.


\bibliographystyle{jplain}
\begin{thebibliography}{9}  
\bibitem{bibi1}
  高知県のいち動物公園協会 業務に関する資料, \url{http://www.noichizoo.or.jp/noichi_hp/gyoumu.html}, 2017年10月13日アクセス

\end{thebibliography}

\end{document}

